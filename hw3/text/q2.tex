
\subsection{Inner Product with Random String}
Let:
\[
	<a,b>=(\sum_ia_ib_i)\%2
\]\[
	L=\{0,1\}^n
\]
Proof:
Let $b\in L\setminus\{0^n\}$. Let $j$ be the first non-zero index of $b$. Define:
\[
	f(a)=a_1a_2...\bar{a_j}..a_n	
\]
Lemma I; $<a,b>\eq_2 <f(a),b>+1$:
\[
	<a,b>\eq_2\sum_ia_ib_i\eq_2 \sum_{i\neq j}a_ib_i+a_jb_j
	\eq_2 \sum_{i\neq j}a_ib_i+a_j
\]\[
	\eq_2 \sum_{i\neq j}a_ib_i+\bar{a_j}+1
	\eq_2 \sum_if(a)_ib_i+1
	<f(a),b>+1
\]
Lemma II; $f$ is bijection $\{a:<a,b>\eq_21\}\longleftrightarrow\{a:<a,b>\eq_20\}$:\\
\[
	x\in\{a:<a,b>\eq_20\}\Leftrightarrow 
	<x,b>\eq_20\Leftrightarrow 
	<f(x),b>+1\eq_20 
\]\[
	\Leftrightarrow
	<f(x),b>\eq_21\Leftrightarrow 
	f(x)\in\{a:<a,b>\eq_21\}
\]
Thus $f$ is bijection.\\
Proof using the Lemma II: Let:
\[
	a=|\{a:<a,b>\eq_21\}|,
	b=|\{a:<a,b>\eq_20\}|
\]
From Lemma II: $a=b$.\\
In addition one of the two cases must always be correct, hence: $a+b=1$.\\
Solving these two equations gives $a=b=\frac{1}{2}$.\\
By definition:
\[
	a=\pr_{a\leftarrow L}[<a,b>=\eq_2 0],
	b=\pr_{a\leftarrow L}[<a,b>=\eq_2 1]	
\]
meaning they are both $\frac{1}{2}$.

\subsection{Inner Product is Universal}
Denote: $H=\{h_a:a\in L\}$.\\
Let $a\in L$.
We want to show that:
\[
	\forall x,y\in L:x\neq y
	\pr_{h\leftarrow H}[h(x)=h(y)]\leq \frac{1}{2}
\]
Proof:\\
Let $x,y\in L:x\neq y$.
\[
	\pr_{h\leftarrow H}[h(x)=h(y)]=
	\pr_{a\leftarrow L}[h_a(x)=h_a(y)]
\]\[
	=\pr_{a\leftarrow L}[h_a(x)=h_a(y)]
	=\pr_{a\leftarrow L}[<a,x>=<a,y>]
\]\[
	=\sum_{b\in\{0,1\}}\pr_{a\leftarrow L}[<a,x>=<a,y>:<a,x>=b]\cdot\pr[<a,x>=b]	
\]\[
	=\sum_{b\in\{0,1\}}\pr_{a\leftarrow L}[<a,x>=<a,y>]\cdot\frac{1}{2}	
	=\sum_{b\in\{0,1\}}\frac{1}{2}\cdot\frac{1}{2}
	=\frac{1}{2}
\]
Since $\frac{1}{2}$ is the size of the output space, this indeed shows that $H$ is
universal hash function family.

\subsection{Purifying Randomness}
Denote $L_n=\{0,1\}^n$.\\
Denote a uniform distribution over $G$ with $\mathbb{U}(G)$.\\
Denote $A\circ B$ operator here denotes a distribution obtained
by sampling from $a\leftarrow A, b\leftarrow B$ and resulting
with the sample $a(b)$.\\
Denote $(A,B)$ to mean a distribution that is sampled
by sampling $a\leftarrow A, b\leftarrow B$ and returning
the sample $(a,b)$.\\

The distribution $<r,<r,s>>$ from the question
can be described with: $(\mathbb{U}(L_n),\mathbb{U}(H)\circ \mathbb{U}(L_1))$.\\
So we want to show that: 
\[
	SD(
		(\mathbb{U}(L_n),\mathbb{U}(H)\circ \mathbb{U}(L_1))
		,(\mathbb{U}(L_{n+1}))
	)=O(\sqrt{\frac{1}{|S|}})
\]

Let $H=\{h_a\}_{a\in L}$ where each $h_a$ is the function 
defined in the last section.\\
Now consider \href{https://en.wikipedia.org/wiki/Leftover_hash_lemma}
{the leftover hash lemma}; indeed we have seen in the prior section,
$H$ is a universal hash function family. Thus we conclude:
\[
	SD((\mathbb{U}(H),\mathbb{U}(H)\circ \mathbb{U}(S)), (\mathbb{U}(H)), \mathbb{U}(L_1))
	\leq\frac{\sqrt{\frac{1}{|S|}}}{2}
\]

Furthermore; $H$ is defined as $\{h_a:a\in L_n\}$, so sampling
$h$ uniformly from $H$ is like sampling $a$ uinformly from $L_n$. So we could
say that:
\[
	(\mathbb{U}(L_n),\mathbb{U}(L_1))
	=(\mathbb{U}(H),\mathbb{U}(L_1))
\]
\[
	(\mathbb{U}(L_n),\mathbb{U}(L_1)\circ \mathbb{U}(S))
	=(\mathbb{U}(H),\mathbb{U}(L_1)\circ \mathbb{U}(S))
\]

Going back to the prior formula:
\[
	SD((\mathbb{U}(L_n),\mathbb{U}(L_1)\circ \mathbb{U}(S))
	,(\mathbb{U}(L_n),\mathbb{U}(L_1)))
	\leq \frac{\sqrt{\frac{1}{|S|}}}{2}
\]

Finally, since sampling $n+1$ bits uniformly is the same
sapling $n$ bits, than sampling another bit:
\[\mathbb{U}(L_{n+1})=(\mathbb{U}(L_n), \mathbb{U}(L_1))\]
And we have the original statment we wanted to prove:
\[
	SD(
		(\mathbb{U}(L_n),\mathbb{U}(H)\circ \mathbb{U}(L_1))
		,(\mathbb{U}(L_{n+1}))
	)=O(\sqrt{\frac{1}{|S|}})
\]

\subsection{Commitments}
