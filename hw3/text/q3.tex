\subsection{Equivalence to Factoring}
We show a there exists
a polynomial turing machine for deciding $L$ iff there exists a polynomial factoring
TM by showing:
\begin{enumerate}
	\item Polynomial TM for factoring which uses a polynomial TM for deciding $L$.
	\item Polynomial TM for deciding $L$ which uses a polynomial TM for factoring.
\end{enumerate}
Armed with the church turing thesis; we describe these two turing machines
as code of a contemporary programming language.

\begin{enumerate}
	\item Factoring given decision:
	\lstinputlisting[language=Python]{code/q3_factor.py}
	Indeed this implementation does a polynomial (w.r to \#bits) number of calls to the oracle,
	and executes only a polynomial number of operations itself:\\
	note how each iteration of the binary search algorithm is polynomial with the number of
	bits of $N$ since it is logarithmic with the size of the search space.
	In addition, there can only be a polynomial (w.r to \#bits) number of factors
	since each one is at-least 2; this means there are only a polynomial number
	of iterations in the main loop.
	\item Decision given factoring:
	\lstinputlisting[language=Python]{code/q3_decide.py}
\end{enumerate}

\subsection{coNP}
Here we show $L\in NP\cap coNP$ in two parts:
\begin{itemize}
	\item $L=L_a\in NP$.
	\item $\bar{L}=L_b\in NP$.
\end{itemize}
To show each of these we will define a relation $R$ which will satisfy all
conditions:
\begin{enumerate}
	\item $\forall(x,y)\in R, |y|=Poly(|x|)$
	\item $(x,y)\in R \Leftrightarrow x\in L$
	\item $\exists M_R, \text{ NTM which decides } R \text{ and is polynomial}$.
\end{enumerate}
\begin{itemize}
	\item Define $R_a$ for $L_a$:
	\[R_a=\{((N,M),p):p\in\mathbb{P}\wedge p>M\wedge N\%p=0\}\]
	It is easy to see that if $(N,M),p\in R_a$ then since $p$ is a prime factor
	of $N$ which is larger than $N$ - it means that $(N,M)\in\mathbb{P}$.
	In addition, $\#bits(p)\leq\#bits(N)$ thus the length is linear (and polynomial).\\
	As for the last condition, we define a NTM $M_a$ which will decide $R_a$:\\
	Given some input $M_a$ will simply check for the three conditions for the pair
	of inputs to be contained within $R_a$. Checking for the first conditon in polynomial
	time is far from trivial, nevertheless it is widely known that $PRIME\in\mathbb{P}$,
	and thus it is possible to check if $p\in\mathbb{P}$ in polynomial time;
	ofcourse it is possible for the other two conditions also.
	
	\item Define $R_b$ for $L_b$:
	\[\{((N,M),S):\prod_{p\in S}p=N\wedge S\subseteq \mathbb{P}\wedge max(S)\leq M\}\]
	Note how $S$ is the set of prime factors of $N$.

	We define a NTM to decide $R_b$:\\
	Given an input $((N,M),S)$ - the machine will check each condition for being
	contained in $R_b$ and accept iff all three are statisfied; taking the product 
	of a set of numbers can comparing it can be done in polynomial time, same as taking the
	maximal value in a list. As before, we know that checking if a number is prime
	can be done in polynomial time, and there can only be a polynomial number of elements in
	the input to begin with (w.r to the size of the input...).
	
	Additionally, it is worth noting how in this case too the length of $S$ will be
	polynomially bound to the length of $N$ (hence to the length of $(N,M)$ too):
	This is because the number of prime factors of $N$ is $O(log(N))$,
	and each of them has length of $O(log(N))$.
\end{itemize}

To sum up, we have shown that $L\in NP$,
and that $L_b=\bar{L}\in NP$ hence $L\in coNP$,
meaining $L\in NP\cap coNP$.