%\href{url}{txt}
\subsection{QR given factorization}
For any $X\in\mathbb{N}$ denote: $QR(X)$ as the set of quadratic residues modulus $X$.\\
Lemma I;
\[
	\forall P,Q\in\mathbb{P}:QR(N)=QR(P)\cap QR(Q)
\]
Proof Lemma I:\\
Let $a\in QR(N)$.
\[
	\exists x,K: x^2\eq_N a
	\Rightarrow x^2=a+(K)PQ
\]\[
	\Rightarrow x^2=a+(KP)Q\wedge x^2=a+(KQ)P
\]\[
	\Rightarrow x^2\eq_Qa\wedge x^2\eq_Pa
\]\[
	\Rightarrow a\in QR(Q)\wedge a\in QR(P)
	\Rightarrow a\in QR(Q)\cap QR(P)
\]
Let $a\in QR(P)\cap QR(Q)$.
Thus:
\[
	\exists x_1,x_2: a\eq_Px_1\wedge a\eq_Qx_2	
\]
Thanks to the
\href{https://en.wikipedia.org/wiki/Chinese_remainder_theorem}
{Chinese remainder theorem}
we know there exists a solution $x$ which satisfies:
\[
	x\eq_Px_1,x\eq_Qx_2
\]
Thus:
\[
	\Rightarrow a\eq_Px^2,a\eq_Qx^2
	\Rightarrow a\eq_{PQ}x^2
	\Rightarrow a\in QR(N)
\]

Now we use the correctness of the Lemma I to define a polynomial algorithm:
\lstinputlisting[language=Python]{code/qr_given_factor.py}

Indeed the names of the variables at lines 4 and 5 are informative (and correct) due
to the properties of Euiler's criterion as seen in class, meaining that
$a$ is qr modulus $P$ iff $a^{(P-1)/2}\eq_P1$, and same with $Q$.\\
This together with Lemma I proves the correctness of this algorithm.\\

We have seen in the last homework how modulus exponantiation can be done efficiently;
which makes this algorithm polynomial.

\subsection{Generating QR}
In the following, all expressions and operations are in the $\mathbb{Z}_N^*$ group
unless said otherwise.\\
Let $x\in QNR(N)$.\\

Lemma I; $x,z\in QNR(N)\Rightarrow zx^{-1}\in QR(N)$:\\
Since $QNR(N)=QNR(P)\cap QNR(Q)$ we get:
\[
	(zx^{-1})^{\frac{p-1}{2}}
	\eq_P(z)^{\frac{p-1}{2}}(x^{-1})^{\frac{p-1}{2}}
	\eq_P(z)^{\frac{p-1}{2}}(x^{\frac{p-1}{2}})^{-1}
	\eq_P(-1)(-1)^{-1}=1
\]
and in the same way under modulus $Q$ we get $(zx^{-1})^{\frac{p-1}{2}}\eq_Q1$.
From Euile's criterion we get that $(zx^{-1})^{\frac{p-1}{2}}\in QR(N)$.\\

Lemma II; $\{y^2x:y\in\mathbb{Z}_N^*\}\supseteq QNR(N)$:\\
Let $z\in QNR(N)$, from Lemma I we get $x^{-1}z\in QR(N)$
Thus:
\[
	\exists y: y^2=zx^{-1}
	\Rightarrow y^2x=z
	\Rightarrow z\in\{y^2x:y\in\mathbb{Z}_N^*\}	
\]

Lemma III; $\{y^2x:y\in\mathbb{Z}_N^*\}\subseteq QNR(N)$:\\
Let $y\in\mathbb{Z}_N^*$. Assume $\exists z:z^2=y^2x$.
Thus:
\[
	z^2y^{-2}=x\Rightarrow (zy^{-1})^2=x\Rightarrow x\notin QNR(N)
\]
Hence the assumption is incorrect, and $y^2x\in QR(N)$.\\

Proof:\\
Define $g(z)=z\cdot x$, $g:QR(N)\longrightarrow QNR(N)$.\\
From Lemma II and III, we get that $g$'s image is exactly $QNR(N)$.\\
$g$ is invertible and thus is a bijection.\\
Let:
\[
	a^2\in QR(N),\;y\aprx^\$\mathbb{Z}_N^*
\]
Since $y^2$ has four different roots:
\[
	\pr_{y\leftarrow \mathbb{Z}_N^*}[y^2=a^2]
	=\frac{4}{|\mathbb{Z}_N^*|}
	=\frac{1}{\frac{|\mathbb{Z}_N^*|}{4}}
	=\frac{1}{|QR(N)|}
\]
\[
	\Rightarrow y^2\aprx^\$QR(N)
\]
And since $g$ is bijection $QR(N)\rightarrow QNR(N)$:
\[
	g(y^2)=y^2x\aprx^\$ QNR(N)
\]
\[\square\]

\subsection{Public-key Encryption}
