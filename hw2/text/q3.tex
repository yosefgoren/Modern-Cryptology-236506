\setcounter{section}{3}
\subsection{}
The problem with the scheme proposed is that Bob can just
fake his result to get whatever he wants: Say that Alice and Bob agreed
that a result of zero means Alice gets the dog and a result of one means
Bob gets the dog, and we also assume that Bob wants the dog.
Bob can get the bit $a$ from Alice, and send her $b=a\oplus 1$.\\
Now both Bob and Alice will see:
\[
	result=b\oplus a = a\oplus 1\oplus a = 1
\]
Which means Bob always gets the dog.

\subsection{}
Alice and Bob can use a coin flipping scheme by deciding that a result of
zero means that Alice gets the dog, and otherwise - Bob does.\\

We could build a coint flipping scheme with the following two algorithms
for Alice (denoted with $A$) and Bob (denoted with $B$):\\
\begin{verbatim}
	A(n):
		b = random
		r = random
		send commit(b, 1^n;r)
		recv c'
		send b,r
		recv b',r'
		if commit(b',1^n;r') != c', return 1
		return xor(b,b')

	B(n):
		b' = random
		r' = random
		recv c
		send commit(b', 1^n;r')
		recv b,r
		send b',r'
		if commit(b,1^n;r) != c, return 0
		return xor(b,b')
\end{verbatim}

Note:\\
Here $send$ and $recv$ are "blocking" or "synchronous" - meaning
the algorithm will not advance to the next statment before verifying
that communication has completed successfully.\\

Correctness:\\
If both follow the protocol, both $b$ and $b'$ are sampled
randomly, and also, both parties do not go into the "if" statment
and thus return $b\oplus b'$; with these two uniformly sampled, their
xor will also be unfiormly sampled, which means both parties return some
value randomly sampled from $\{0,1\}$.\\

Alice Soundness:\\
If alice follows the protocol, she will only send her bit after she 
sees a commit from her opponent. So her opponent cannot know the value
of $b$ before sending her some commit $c$. This means the opponent cannnot
wait until he knows $b$ to decide the value of $b'$ (knowing $b$ would mean breaking
the "binding" property of the commitment scheme).\\
If so, there cannot be any dependence between the variables
$b$ and $b'$ - meaning that regardless of the $b'$ distribution: $b\oplus b'$ is distributed
evenly between zero and one.\\
So we can conclude that in cases where the opponent decides
to send a legal commit message, Alice will return zero with probability $\frac{1}{2}$.\\

Furthermore, in cases where the opponent does not send a legal commit;message pair - Alice
finds this is the case and returns zero.\\

To sum up, there are two cases; if a legal commit is sent: $\pr[A(n)=1]=\frac{1}{2}$.
Otherwise: $\pr[A(n)=1]=0$.\\
Regardless of the distribution between these two cases, the total
probability for $A(n)=1$, is no more than $\frac{1}{2}$.\\

Bob Correctness:\\
Mirror proof. Indeed Bob also requires knowing Alice's commit message
before sending $b'$ in the same way Alice waits for Bob's commit before
sending $b$ - and in addition, Bob also checks the correctness of the commit
message sent to him in the same way.

\subsection{}
Indeed both can exploit an opponent following his/her part of the protocol
to cause the other to sample their own token - by simply sending an incorrect
commit message. For example, if Bob sends an incorrect commit message,
Alice will check that at the end - and thus return 0. If Bob is also
sampling 0 himself, both sides have "agreed" that the dog should go to Alice.
