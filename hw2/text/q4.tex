\subsection{}
Trivial aglorithm:
\begin{lstlisting}[language=Python]
	def ForgeTag(reference_msg, reference_tag, new_msg):
	result = ""
	for i in range(len(reference_tag)):
		result += random([0, 1])
	return result
\end{lstlisting}

Correctness:\\
Denote correct tag for $new\_msg$ as $new\_tag$.
Since the length of $new\_tag$ is $t$ and the algorithm uniformly samples
from all binary strings of length $t$; there is a $(\frac{1}{2})^t$
chance that the string randomized by the algorithm will be $new\_tag$.
Meaning this is the probability of the algorithm being correct.

\subsection{}
Pairwise Independence:\\
Let $x_1,x_2\in\mathbb{F}: x_1\neq x_2$. Let $y_1,y_2\in\mathbb{F}$.\\
\[
	=\pr_{h\leftarrow H}[h(x_1)=y_1\wedge h(x_2)=y_2]
	=\pr_{a,b\leftarrow\mathbb{F}}[ax_1+y_1=b\wedge ax_2+y_2=b]\]\[
	=\sum_{B\in\mathbb{F}}\pr_{a,b\leftarrow\mathbb{F}} [ax_1+y_1=b\wedge ax_2+y_2=b\mid b=B]\pr_{b\leftarrow\mathbb{F}}[b=B]\]\[
	=\sum_{B\in\mathbb{F}}\pr_{a,b\leftarrow\mathbb{F}} [ax_1+y_1=B\wedge ax_2+y_2=B]\frac{1}{|\mathbb{F}|}\]\[
	=\frac{1}{|\mathbb{F}|}\sum_{B\in\mathbb{F}}\pr_{a,b\leftarrow\mathbb{F}} [ax_1+y_1=ax_2+y_2=B]\]\[
	=\frac{1}{|\mathbb{F}|}\pr_{a,b\leftarrow\mathbb{F}} [ax_1+y_1=ax_2+y_2]
	=\frac{1}{|\mathbb{F}|}\pr_{a,b\leftarrow\mathbb{F}} [a(x_1-x_2)=y_2-y_1]\]\[
	\eq^{(*)}\frac{1}{|\mathbb{F}|}\pr_{a,b\leftarrow\mathbb{F}} [a=(y_2-y_1)((x_1-x_2))^{-1}]
	=\frac{1}{|\mathbb{F}|^2}
	=\frac{1}{2^{2m}}
	=2^{-2m}
\]
\[\square\]

\[
	(*): x_1\neq x_2\Rightarrow x_1-x_2\neq 0\Rightarrow (x_1-x_2)^{-1}\in\mathbb{F}
\]

The general case for the range field:\\
To make use of this same hash family (and proof of independence), in a setting as to
transfer items from $\mathbb{F}_1$ to $\mathbb{F}_2$; define:
\[
	H=\{h_{a,b}\mid a'b\in\mathbb{F}_2\}; h_{a,b}(x)=a\cdot T(x)+b
\]
Where $T(\cdot)$ is some transformation $T:\mathbb{F}_1\longrightarrow \mathbb{F}_2$, such that
$T(0_{\mathbb{F}_1})=0_{\mathbb{F}_2}$ and $T(x+1_{\mathbb{F}_1})=T(x)+T(1_{\mathbb{F}_1})$.
Under these assumtions, the provided proof from before is correct in the general case.\\
Also note how in the prior proof, the determining factor for the final probability was
the size of the field from which $a,b$ are sampled, i.e. $\mathbb{F}_2$ in the 
general case. Meaning the probability is indeed $2^{-2m}$, where $m=|\mathbb{F}_2|$.

\subsection{}
Define the following MAC algorithm:
\begin{lstlisting}[language=Python]
	def h(a, b, x):
		"""This is the hash function parameterized by 'a,b' from the H
			family of pairwise independent functions (from section 4.2).
			The output field size is the same as that of 'x' and 'y'."""
		...

	def MakeTag(key, msg):
		mid = len(key)//2
		kL = key[:mid]
		kR = key[mid:]
		return h(kL, kR, msg)
\end{lstlisting}

Correctness:\\
Given that the key is selected randomly, (and the opponent does not have access to it),
Both sides of the key are also selected randomly - meaning that the algorithm 
has simply sampled as random hash function from $H$. As stated in section 4.2;
this means that the behaviour of this randomly selected hash from $H$ must be
the same as that of a completely random function over the same space. If indeed 
our tagging scheme behaves in the same way as a random function over the same sapce,
it means that any PPT will not be able to deduce anything from just one output 
of our tagging scheme (with a spesific key). Thus any opponent algorithm
must not do any better than either guessing an output; which would yeild that
algorithm a chance of success equal to $\frac{1}{N}$ where $N$ is the size of the
output domain. In our case, $N=2^t$, meaning no opponent can do better than $2^{-t}$.